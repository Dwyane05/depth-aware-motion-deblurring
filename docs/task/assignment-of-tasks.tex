\documentclass[a4paper, 12pt]{scrartcl}


% UTF-8 encoding
\usepackage[utf8]{}
\usepackage{fontspec}
\usepackage{csquotes}
\usepackage{hyperref}
\pagenumbering{gobble}% Remove page numbers (and reset to 1)

\begin{document}
\section*{Assignment of Tasks - Study Thesis}
\begin{description}
\item[Task:] \enquote{Depth-Aware Motion Deblurring in Stereo Images}
\\
\item[Name:] Franziska Krüger
\item[Course of Studies:] Computer Science (Diploma)  \hfill{\textbf{Mat. Nr.:} 3858788}
\item[E-mail:] Franziska.Krueger1@tu-dresden.de
\\
\item[Tutor:] Dr.-Ing. Anita Sellent
\item[Professor:] Prof. Carsten Rother
% the study thesis should take about 20 weeks with over all 270 hours of work (13,5 h/week)
\item[Start:] 01.11.2015  \hfill{\textbf{End:} 21.03.2016} \hfill{\textbf{Handed in:} ...............}
\end{description}



\subsection*{Motivation and Goal}
A lot of pictures are taken with a mobile phone or a hand-held camera. Therefore blur is a widely spread problem. This blur mostly occur due to the shaking of the camera during the exposure. There exists several algorithms for removing such blur from single images or stereo image pairs. The later is interesting because of the additional depth information that can be computed and used for a better deblurring. Furthermore the necessary hardware to obtain stereo image pairs is more and more available – even in mobile phones.

A very interesting iterative algorithm for motion deblurring with stereo images was developed by Xu and Jia. It uses a spatially-varying point spread functions (PSF) to deblur the image on each depth level. For further improvements of this algorithm a reference implementation is needed. This is also useful to get a better understanding of the difficult occlusion handling.

The goal for this study thesis is a reference implementation and an evaluation of the results. An optional goal are further improvements of the PSF estimation.



\subsection*{Tasks}
\begin{itemize}
\item Literature research on deblurring - especially on stereo motion deblurring
\item Reference implementation of the depth aware stereo deblurring approach from Xu and Jia
\item Evaluation of this algorithm on pictures taken with a stereo camera
\item Optional: improvement of the PSF estimation with depth-dependent assumptions \\(PSF is scaled between each depth layer)
\end{itemize}



\subsection*{References}
\textbf{Xu L., Jia J.}: Depth-Aware Motion Deblurring \\
\url{http://www.cse.cuhk.edu.hk/leojia/papers/depth_deblur_iccp12.pdf}

% \begin{tabular}{lp{2em}l}
%  \hspace{5cm}   && \hspace{5cm} \\\cline{1-1}\cline{3-3}
%  Signature of Student     && Signature of Professor
% \end{tabular}

\end{document}